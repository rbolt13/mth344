% Options for packages loaded elsewhere
\PassOptionsToPackage{unicode}{hyperref}
\PassOptionsToPackage{hyphens}{url}
%
\documentclass[
]{article}
\usepackage{amsmath,amssymb}
\usepackage{lmodern}
\usepackage{iftex}
\ifPDFTeX
  \usepackage[T1]{fontenc}
  \usepackage[utf8]{inputenc}
  \usepackage{textcomp} % provide euro and other symbols
\else % if luatex or xetex
  \usepackage{unicode-math}
  \defaultfontfeatures{Scale=MatchLowercase}
  \defaultfontfeatures[\rmfamily]{Ligatures=TeX,Scale=1}
\fi
% Use upquote if available, for straight quotes in verbatim environments
\IfFileExists{upquote.sty}{\usepackage{upquote}}{}
\IfFileExists{microtype.sty}{% use microtype if available
  \usepackage[]{microtype}
  \UseMicrotypeSet[protrusion]{basicmath} % disable protrusion for tt fonts
}{}
\makeatletter
\@ifundefined{KOMAClassName}{% if non-KOMA class
  \IfFileExists{parskip.sty}{%
    \usepackage{parskip}
  }{% else
    \setlength{\parindent}{0pt}
    \setlength{\parskip}{6pt plus 2pt minus 1pt}}
}{% if KOMA class
  \KOMAoptions{parskip=half}}
\makeatother
\usepackage{xcolor}
\IfFileExists{xurl.sty}{\usepackage{xurl}}{} % add URL line breaks if available
\IfFileExists{bookmark.sty}{\usepackage{bookmark}}{\usepackage{hyperref}}
\hypersetup{
  pdftitle={MTH 344-001 Winter 2022 HW 1},
  pdfauthor={Randi Bolt},
  hidelinks,
  pdfcreator={LaTeX via pandoc}}
\urlstyle{same} % disable monospaced font for URLs
\usepackage[margin=1in]{geometry}
\usepackage{graphicx}
\makeatletter
\def\maxwidth{\ifdim\Gin@nat@width>\linewidth\linewidth\else\Gin@nat@width\fi}
\def\maxheight{\ifdim\Gin@nat@height>\textheight\textheight\else\Gin@nat@height\fi}
\makeatother
% Scale images if necessary, so that they will not overflow the page
% margins by default, and it is still possible to overwrite the defaults
% using explicit options in \includegraphics[width, height, ...]{}
\setkeys{Gin}{width=\maxwidth,height=\maxheight,keepaspectratio}
% Set default figure placement to htbp
\makeatletter
\def\fps@figure{htbp}
\makeatother
\setlength{\emergencystretch}{3em} % prevent overfull lines
\providecommand{\tightlist}{%
  \setlength{\itemsep}{0pt}\setlength{\parskip}{0pt}}
\setcounter{secnumdepth}{-\maxdimen} % remove section numbering
\ifLuaTeX
  \usepackage{selnolig}  % disable illegal ligatures
\fi

\title{MTH 344-001 Winter 2022 HW 1}
\author{Randi Bolt}
\date{1/14/2022}

\begin{document}
\maketitle

\hypertarget{do-the-following-define-operations-on-the-given-set-why-or-why-not-your-answers-only-need-to-be-a-sentance-or-two-if-these-do-define-operations-you-do-not-need-to-check-if-theyre-commutative-associatve-ect.-in-this-problem.}{%
\subsection{1. Do the following define operations on the given set? Why
or why not? (Your answers only need to be a sentance or two; if these do
define operations, you do not need to check if they're commutative,
associatve, ect. in this
problem.)}\label{do-the-following-define-operations-on-the-given-set-why-or-why-not-your-answers-only-need-to-be-a-sentance-or-two-if-these-do-define-operations-you-do-not-need-to-check-if-theyre-commutative-associatve-ect.-in-this-problem.}}

\hypertarget{a-ab4a-b-on-mathbbz}{%
\subsubsection{\texorpdfstring{(a) \(a*b=4a-b\) on
\(\mathbb{Z}\)}{(a) a*b=4a-b on \textbackslash mathbb\{Z\}}}\label{a-ab4a-b-on-mathbbz}}

Yes, a*b should work for any two elements in \(\mathbb{Z}\), and produce
a unique output that exists in \(\mathbb{Z}\).

\hypertarget{b-ab3b-on-mathbbq}{%
\subsubsection{\texorpdfstring{(b) \(a*b=3^b\) on
\(\mathbb{Q}\)}{(b) a*b=3\^{}b on \textbackslash mathbb\{Q\}}}\label{b-ab3b-on-mathbbq}}

No, because an operation needs two elements a and b, and a*b only
performs an operation on one element.

\hypertarget{c-abfracaab2-on-mathbbr}{%
\subsubsection{\texorpdfstring{(c) \(a*b=\frac{a}{ab+2}\) on
\(\mathbb{R}\)}{(c) a*b=\textbackslash frac\{a\}\{ab+2\} on \textbackslash mathbb\{R\}}}\label{c-abfracaab2-on-mathbbr}}

No, because it is possible for \(ab=-2\) which would make a*b undefined.

\hypertarget{d-abfrac1ab2-on-mathbbr}{%
\subsubsection{\texorpdfstring{(d) \(a*b=\frac{1}{ab+2}\) on
\(\mathbb{R}^+\)}{(d) a*b=\textbackslash frac\{1\}\{ab+2\} on \textbackslash mathbb\{R\}\^{}+}}\label{d-abfrac1ab2-on-mathbbr}}

Yes, a*b should work for any two elements in \(\mathbb{R}^+\), and
product a unique output that also exists in \(\mathbb{R}^+\).

\hypertarget{define-an-operation-on-mathbbr-by-ab3a-4b2}{%
\subsection{\texorpdfstring{2. Define an operation * on \(\mathbb{R}\)
by
\(a*b=3a-4b+2\)}{2. Define an operation * on \textbackslash mathbb\{R\} by a*b=3a-4b+2}}\label{define-an-operation-on-mathbbr-by-ab3a-4b2}}

\hypertarget{a-is-commutative}{%
\subsubsection{(a) Is * commutative?}\label{a-is-commutative}}

No, it is given that \(a*b=3a-4b+2\), but \(b*a=3b-4a+2\), which is
clearly not the same.

To show by this by example let \(a=1\), and \(b=2\). We see that
\(a*b=3a-4b+2=3(1)-4(2)+2=-3\) and \(b*a=3b-4a+2=3(2)-4(1)+2=4\).

Since \(a*b\ne b*a\), then * is NOT commutative. \(\quad\quad\square\)

\hypertarget{b-is-associative}{%
\subsubsection{(b) Is * associative?}\label{b-is-associative}}

No,

\begin{equation}
\label{1b}
\begin{split}
(a*b)*c &= (3a-4b+2)*c\\
&= 3(3a-4b+2)-4c+2\\
&= 9a-12b+6-4c+2\\
&= 9a-12b-4c+8
\end{split}
\end{equation}

\begin{equation}
\label{1b-2}
\begin{split}
a*(b*c) &= a*(3b-4c+2)\\
&= 3a-4(3b-4c+2)+2\\
&= 3a-12b+16c-8+2\\
&= 3a-12b+16c-6
\end{split}
\end{equation}

Since \((a*b)*c\ne a*(b*c)\), then * isn't associative.
\(\quad\quad\square\)

\hypertarget{c-is-there-an-identity-element-ein-mathbbr-w.r.t.}{%
\subsubsection{\texorpdfstring{(c) Is there an identity element
\(e\in \mathbb{R}\) w.r.t.
*?}{(c) Is there an identity element e\textbackslash in \textbackslash mathbb\{R\} w.r.t. *?}}\label{c-is-there-an-identity-element-ein-mathbbr-w.r.t.}}

No, let \(e\in\mathbb{R}\) such that \(e*a=a\forall a\in\mathbb{R}\).

\begin{equation}
\label{1c}
\begin{split}
e*a=a\Rightarrow 3a-4e+2 &= a \\
&\Rightarrow 4e = 2a+2\\
&\Rightarrow e = \frac{1}{4}(2a+2)= \frac{1}{2}a+\frac{1}{2}
\end{split}
\end{equation}

Since the identity element must be constant then there is no identity in
\(\mathbb{R}\) w.r.t. *. \(\quad\quad\square\)

\hypertarget{d-does-every-element-ainmathbbr-have-an-inverse-w.r.t.}{%
\subsubsection{\texorpdfstring{(d) Does every element \(a\in\mathbb{R}\)
have an inverse w.r.t.
*?}{(d) Does every element a\textbackslash in\textbackslash mathbb\{R\} have an inverse w.r.t. *?}}\label{d-does-every-element-ainmathbbr-have-an-inverse-w.r.t.}}

No, since there is no identity then are no inverse in \(\mathbb{R}\)
w.r.t. *. \(\quad\quad \square\)

\hypertarget{define-an-operation-on-mathbbr-by-abab-ab.}{%
\subsection{\texorpdfstring{3. Define an operation * on \(\mathbb{R}\)
by
\(a*b=a+b-ab\).}{3. Define an operation * on \textbackslash mathbb\{R\} by a*b=a+b-ab.}}\label{define-an-operation-on-mathbbr-by-abab-ab.}}

\hypertarget{a-is-commutative-1}{%
\subsubsection{(a) Is * commutative?}\label{a-is-commutative-1}}

Yes,

\begin{equation}
\label{3a}
\begin{split}
a*b &= a+b-ab\\
&= b+a-ab\quad\text{ since + is commutative}\\
&= b+a-ba\quad\text{since }\cdot\text{ is commutative}\\
&= b*a
\end{split}
\end{equation}

Since \(a*b=b*a\) then * is commutative. \(\quad\quad\square\)

\hypertarget{b-is-associative-1}{%
\subsubsection{(b) Is * associative?}\label{b-is-associative-1}}

Yes,

\begin{equation}
\label{3b}
\begin{split}
a*(b*c) &= a*(b+c-bc)\\
&= a+(b+c-bc)-a(b+c-bc)\\
&= a+b+c-ab-ac-bc+abc
\end{split}
\end{equation}

\begin{equation}
\label{3b2}
\begin{split}
(a*b)*c &= (a+b-ab)*c\\
&= (a+b-ab)+c-c(a+b-ab)\\
&= a+b-ab+c-ca+cb-abc\\
&= a+b+c-ab-ac-bc+abc
\end{split}
\end{equation}

Since \(a*(b*c)=(a*b)*c\) then * is associative. \(\quad\quad\square\)

\hypertarget{c-is-there-an-identity-element-ein-mathbbr-w.r.t.-1}{%
\subsubsection{\texorpdfstring{(c) Is there an identity element
\(e\in \mathbb{R}\) w.r.t.
*?}{(c) Is there an identity element e\textbackslash in \textbackslash mathbb\{R\} w.r.t. *?}}\label{c-is-there-an-identity-element-ein-mathbbr-w.r.t.-1}}

Yes, since \(0\in\mathbb{R}\) \(a*0=0*a=a+0-a(0)=a\). Therefore there is
an identity element \(e=0\) w.r.t. *. \(\quad\quad\square\)

\hypertarget{d-does-every-element-ainmathbbr-have-an-inverse-w.r.t.-1}{%
\subsubsection{\texorpdfstring{(d) Does every element \(a\in\mathbb{R}\)
have an inverse w.r.t.
*?}{(d) Does every element a\textbackslash in\textbackslash mathbb\{R\} have an inverse w.r.t. *?}}\label{d-does-every-element-ainmathbbr-have-an-inverse-w.r.t.-1}}

Yes, suppose \(b=a^{-1}\), then \(a*b=b*a=0\) (Identity found if part
c).

\begin{equation}
\label{3d}
\begin{split}
a*b=0 &\Rightarrow a+b-ab=0\\
&\Rightarrow b(1-a)=-a\\
&\Rightarrow b=\frac{-a}{1-a}
\end{split}
\end{equation}

Therefore \(\forall a\in\mathbb{R}\) except for 1 has an inverse given
by: \(a^{-1}=\frac{-a}{1-a}\). \(\quad\quad\square\)

\hypertarget{define-an-operation-on-the-set-gxyinmathbbrtimesmathbbrxne-0-by-abcdacadbc.-prove-that-langle-grangle-is-an-abelian-group.-you-do-not-need-to-check-associativity.-we-will-show-this-in-class.}{%
\subsection{\texorpdfstring{4. Define an operation * on the set
\(G=\{(x,y)\in\mathbb{R}\times\mathbb{R}|x\ne 0\}\) by
\[(a,b)*(c,d)=(ac,ad+bc).\] Prove that \(\langle G,*\rangle\) is an
abelian group. (You do not need to check associativity. We will show
this in
class.)}{4. Define an operation * on the set G=\textbackslash\{(x,y)\textbackslash in\textbackslash mathbb\{R\}\textbackslash times\textbackslash mathbb\{R\}\textbar x\textbackslash ne 0\textbackslash\} by (a,b)*(c,d)=(ac,ad+bc). Prove that \textbackslash langle G,*\textbackslash rangle is an abelian group. (You do not need to check associativity. We will show this in class.)}}\label{define-an-operation-on-the-set-gxyinmathbbrtimesmathbbrxne-0-by-abcdacadbc.-prove-that-langle-grangle-is-an-abelian-group.-you-do-not-need-to-check-associativity.-we-will-show-this-in-class.}}

Let \(G=\{(x,y)\in\mathbb{R}\times\mathbb{R}|x\ne0\}\). Define * on G by
\[(a,b)*(c,d)=(ac,ad+bc)\]

For \(\langle G,*\rangle\) is an abelian group it must be commutative,
associative, have an identity element, and each element must have an
inverse.

\begin{enumerate}
\def\labelenumi{(\roman{enumi})}
\tightlist
\item
  Commutative
\end{enumerate}

For \((a,b)*(c,d)\) to be commutative then
\((a,b)*(c,d)=(d,a)*(b,c)=(c,d)*(a,b)=(b,c)*(d,a)\).

Each can be defined:

\((a,b)*(c,d)=(ac,ad+bc)\)

\((d,a)*(b,c)=(db,dc+ab)\)

\((c,d)*(a,b)=(ca,cb+da)\)

\((b,c)*(d,a)=(bd,ba+cd)\)

Since multiplication and addition are communicative on \(\mathbb{R}\),
we can also show that

\begin{enumerate}
\def\labelenumi{(\arabic{enumi})}
\tightlist
\item
  \((a,b)*(c,d)=(ac,ad+bc)=(ca,da+cb)=(ca,cb+da)=(c,d)*(a,b)\)
\end{enumerate}

and

\begin{enumerate}
\def\labelenumi{(\arabic{enumi})}
\setcounter{enumi}{1}
\tightlist
\item
  \((d,a)*(b,c)=(db,dc+ab)=(db,cd+ba)=(db,ba+cd)=(b,c)*(d,a)\)
\end{enumerate}

However (1) \(\ne\) (2).

To show this by example let \(a=1\), \(b=2\) , \(c=3\), and \(d=4\).
Then,

\begin{enumerate}
\def\labelenumi{(\arabic{enumi})}
\item
  \((a,b)*(c,d)=(ac,ad+bc)=(1(3),1(4)+2(3))=(3,10)\)
\item
  \((d,a)*(b,c)=(db,dc+ab)=(4(2),4(3)+1(2))=(8,14)\)
\end{enumerate}

Since \((3,10)\ne (8,14)\) and
\((a,b)*(c,d)=(c,d)*(a,b)\ne (d,a)*(b,c)=(b,c)*(d,a)\) then
\(\langle G,*\rangle\) is not commutative. \(\quad\quad\square\)

\begin{enumerate}
\def\labelenumi{(\roman{enumi})}
\setcounter{enumi}{1}
\tightlist
\item
  Associative
\end{enumerate}

done in class.

\begin{enumerate}
\def\labelenumi{(\roman{enumi})}
\setcounter{enumi}{2}
\tightlist
\item
  Identity
\end{enumerate}

Let \((a,b)*(e_1,e_2)=(a,b)\) for \(e_1,e_2\in\mathbb{R}\).

\begin{equation}
\label{4iii1}
\begin{split}
(a,b)*(e_1,e_2)=(a,b)&\Rightarrow (ae_1,ae_2+be_1)=(a,b) \\
&\Rightarrow (e_1,e_2)=(1,0)
\end{split}
\end{equation}

Checking the other order,

\begin{equation}
\label{4iii2}
\begin{split}
(e_1,e_2)*(a,b)=(1,0)*(a,b) &= (1a,1b+a0) \\
&= (a,b)
\end{split}
\end{equation}

Since \((a,b)*(e_1,e_2)=(a,b)\) there is an identity in \(\mathbb{R}\)
w.r.t. \(\langle G,*\rangle\).\(\quad\quad\square\)

\begin{enumerate}
\def\labelenumi{(\roman{enumi})}
\setcounter{enumi}{3}
\tightlist
\item
  Inverse
\end{enumerate}

Suppose \((c,d)=(a,b)^{-1}\), then \((a,b)*(c,d)=(1,0)\) (identity found
in part iii).

\begin{equation}
\label{4iv}
\begin{split}
(a,b)*(c,d)=(1,0)&\Rightarrow(ac,ad+bc)=(1,0) \\
&\Rightarrow (c,ad)=(\frac{1}{a},-bc)\\
&\Rightarrow (c,a^{-1}ad)=(\frac{1}{a},a^{-1}(-b)c)\\
&\Rightarrow (c,d)=(\frac{1}{a},-ba^{-1}c)\\
&\Rightarrow (c,d)=(\frac{1}{a},-ba^{-1}a^{-1})\\
&\Rightarrow (a,b)^{-1}=(\frac{1}{a},-ba^{-2})
\end{split}
\end{equation}

Therefore \(\forall (a,b)\in\mathbb{R}\), \(\langle G,*\rangle\) has an
inverse given by \((\frac{1}{a},-ba^{-2})\). \(\quad\quad\square\)

By parts ii - iv, G is associative, has an identity element, and every
element has an inverse. Therefore \(\langle G,*\rangle\) is an group,
but because it isn't commutative, then \(\langle G,*\rangle\) is not an
abelian group. \(\quad\quad\square\)

\hypertarget{let-abcx-be-elements-of-a-group-g.-solve-the-following-system-of-equations-for-x}{%
\subsection{5. Let a,b,c,x be elements of a group G. Solve the following
system of equations for x
:}\label{let-abcx-be-elements-of-a-group-g.-solve-the-following-system-of-equations-for-x}}

\hypertarget{a-bx2ax-1quad-and-x4c}{%
\subsubsection{\texorpdfstring{(a) \(bx^2=ax^{-1}\quad\) and
\(x^4=c\)}{(a) bx\^{}2=ax\^{}\{-1\}\textbackslash quad and x\^{}4=c}}\label{a-bx2ax-1quad-and-x4c}}

\(bx^2=ax^{-1}\Rightarrow b=ax^{-3}\Rightarrow a^{-1}b=x^{-3}\)

\(x^4=c\Rightarrow x=cx^{-3}=ca^{-1}b=a^{-1}bc\)

Therefore \(x=a^{-1}bc\). \(\quad\quad\square\)

\hypertarget{b-x2cbxa-1quad-and-xcacax}{%
\subsubsection{\texorpdfstring{(b) \(x^2c=bxa^{-1}\quad\) and
\(xca=cax\)}{(b) x\^{}2c=bxa\^{}\{-1\}\textbackslash quad and xca=cax}}\label{b-x2cbxa-1quad-and-xcacax}}

\begin{equation}
\label{5b}
\begin{split}
x^2c &= bxa^{-1}\\
\Rightarrow xxc &= bxa^{-1}\\
\Rightarrow xxca &= bxa^{-1}a\\
\Rightarrow x(xca) &= bx(a^{-1}a)\\
\Rightarrow x(cax) &= bx(e)\\
\Rightarrow xcaxx^{-1} &= bxx^{-1}\\
\Rightarrow xca(xx^{-1}) &= b(xx^{-1})\\
\Rightarrow xca(e) &= b(e)\\
\Rightarrow xc(aa^{-1}) &=ba^{-1}\\
\Rightarrow x(cc^{-1}) &= ba^{-1}c^{-1}\\
\Rightarrow x &= ba^{-1}c^{-1}
\end{split}
\end{equation}

Therefore \(x=ba^{-1}c^{-1}\). \(\quad\quad\square\).

\hypertarget{this-problem-asks-you-to-consider-the-importance-of-using-proper-notation-for-inverses-in-groups.-your-answer-only-needs-to-be-a-sentence-or-two.}{%
\subsection{6. This problem asks you to consider the importance of using
proper notation for inverses in groups. Your answer only needs to be a
sentence or
two.}\label{this-problem-asks-you-to-consider-the-importance-of-using-proper-notation-for-inverses-in-groups.-your-answer-only-needs-to-be-a-sentence-or-two.}}

\hypertarget{suppose-ab-and-x-are-elements-of-a-nonabelian-group-g-and-that-we-want-to-solve-thte-equation-axb-for-x.-why-would-it-be-incorrect-and-unclear-to-say-that-the-solution-is-xfracba}{%
\subsection{\texorpdfstring{Suppose a,b, and x are elements of a
\emph{nonabelian} group G, and that we want to solve thte equation
\(ax=b\) for x. Why would it be incorrect and unclear to say that the
solution is
\(x=\frac{b}{a}\)}{Suppose a,b, and x are elements of a nonabelian group G, and that we want to solve thte equation ax=b for x. Why would it be incorrect and unclear to say that the solution is x=\textbackslash frac\{b\}\{a\}}}\label{suppose-ab-and-x-are-elements-of-a-nonabelian-group-g-and-that-we-want-to-solve-thte-equation-axb-for-x.-why-would-it-be-incorrect-and-unclear-to-say-that-the-solution-is-xfracba}}

It would be incorrect and unclear to say that \(x=\frac{b}{a}\) for a
\emph{nonabelian} group G, because a \emph{nonabelian} group wouldn't be
commutative. This is important because the side an operation happens on
matters. Meaning that
\((ax=b)\Rightarrow (x=\frac{b}{a})\ne(\frac{a}{b}=x)\Leftarrow (bx=a)\).

\end{document}
